\setcounter{chapter}{2}
\chapter{Counting and Probability}
%Section C.1-----------------------------------------------------------------------
\section{Counting}
\subsection*{Exercises}
\begin{enumerate}[\thesection-1]
%
\item An $n$-string has $(n - k + 1)$ $k$-substrings, where $1 \leq k \leq n$. Therefore, an $n$-string has in total
\[
\sum^n_{k = 1} (n - k + 1) = \frac{1}{2}n(n + 1)
\]
substrings.
%
\item There are $2^{2^n}$ $n$-input, $1$-output boolean functions. There are $2^{m2^n}$ $n$-input, $m$-output boolean functions.
%
\item $(n - 1)!$.
%
\item There are $49$ even numbers and 50 odd numbers in the set $\sete{1, 2, \ldots, 99}$. There are two cases in which the sum of the three distinct numbers chosen from this set is even: \begin{inparaenum}[(1)] \item all are even, \item one is even, the other two are odd. \end{inparaenum} Also, make use of rules of sum and of product.
%
\item Immediate by definition.
%
\item Immediate by definition.
%
\item By the hint and the rule of sum.
%
\item Skipped.
%
\item Consider choosing two numbers $m < M$ among $1, \ldots, n + 1$. For each fixed value of $m$, how many possibilities are there for $M$?
%
\end{enumerate}
%End of Section C.1----------------------------------------------------------------
