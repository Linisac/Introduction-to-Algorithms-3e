%Last modified by Linisac, 03:37AM 04/11/2016
\documentclass{report}

%Packages
\usepackage{titlesec}
\usepackage{clrscode3e}
\usepackage{amssymb}
\usepackage{amsmath}
\usepackage{amsthm}
\usepackage{enumerate}
\usepackage{paralist}
\usepackage{colonequals}
\usepackage{appendix}
\usepackage[margin=1.5cm]{geometry}

%New oommands
%%Styles
\newcommand{\textib}[1]{\textit{\textbf{#1}}}
\newcommand{\df}[2]{{\displaystyle\frac{#1}{#2}}}
%%Enumeration, sequence, chain and serial
\newcommand{\enum}[2]{#1_1, #1_2, \ldots, #1_#2} %enumeration: \enum{a}{n} gives a_1, a_2, \ldots, a_n
\newcommand{\seq}[2]{\langle\enum{#1}{#2}\rangle} %sequence: \seq{a}{n} \gives \langle a_1, a_2, \ldots, a_n \rangle
\newcommand{\chaininfp}[3]{#1 #2 #3 #2\ldots} %primitive infinite chain: \chaininfp{a}{<}{b} gives a < b < \ldots
\newcommand{\chaininf}[2]{\chaininfp{#1_1}{#2}{#1_2}} %infinite chain: \chaininf{a}{<} gives a_1 < a_2 < \ldots
\newcommand{\chain}[3]{\chaininf{#1}{#3}#3#1_#2} %(finite) chain: \chain{a}{n}{<} gives a_1 < a_2 < \ldots < a_n
\newcommand{\chaininc}[2]{\chain{#1}{#2}{<}} %increasing chain: \chaininc{a}{n} gives a_1 < a_2 < \ldots < a_n
\newcommand{\chaindec}[2]{\chain{#1}{#2}{>}} %decreasing chain: \chaindec{a}{n} gives a_1 > a_2 > \ldots > a_n
\newcommand{\chainni}[2]{\chain{#1}{#2}{\geq}} %noninreasing chain: \chainni{a}{n} gives a_1 \geq a_2 \geq \ldots \geq a_n
\newcommand{\chainnd}[2]{\chain{#1}{#2}{\leq}} %nondecreasing chain: \chainnd{a}{n} gives a_1 \leq a_2 \leq \ldots \leq a_n
\newcommand{\serialinfp}[3]{#1 #2 #3 #2\ldots} %primitive infinite serial operation: \serialinfp{a}{+}{b} gives a + b + \ldots
\newcommand{\serialinf}[2]{\serialinfp{#1_1}{#2}{#1_2}} %infinite serial operation: \serialinf{a}{+} gives a_1 + a_2 + \ldots
\newcommand{\serial}[3]{\serialinf{#1}{#3}#3#1_#2} %(finite) serial operation: \serial{a}{n}{+} gives a_1 + a_2 + \ldots + a_n
\newcommand{\etl}{\ldots} %etc, lower
\newcommand{\etc}{\cdots} %etc, centered
%%Mathematical operators notations
\newcommand{\multp}{\cdot} %multiplication
%%Integer functions
\newcommand{\floor}[1]{\left\lfloor #1 \right\rfloor}
\newcommand{\ceil}[1]{\left\lceil #1 \right\rceil}
%%Asymptotic functions notations
\newcommand{\bigoh}[1]{O\left(#1\right)}
\newcommand{\littleoh}[1]{o\left(#1\right)}
\newcommand{\atheta}[1]{\Theta\left(#1\right)} %asymptotic theta
\newcommand{\bigomega}[1]{\Omega\left(#1\right)}
\newcommand{\littleomega}[1]{\omega\left(#1\right)}
%%Set-theoretic notations
\newcommand{\sete}[1]{\left\{#1\right\}} %set by enumeration
\newcommand{\setd}[2]{\left\{#1 : #2\right\}} %set by description
\newcommand{\intersect}{\cap}
\newcommand{\union}{\cup}
\newcommand{\cmplmnt}[1]{\overline{#1}}
\newcommand{\nat}{\mathbb{N}} %the set of natural numbers
\newcommand{\zah}{\mathbb{Z}} %the set of integers
\newcommand{\rat}{\mathbb{Q}} %the set of rational numbers
\newcommand{\real}{\mathbb{R}} %the set of real numbers
\newcommand{\cart}{\times} %cartesian product
\newcommand{\card}[1]{\left|#1\right|} %cardinality
\newcommand{\inv}[1]{#1^{-1}} %inverse (of a function)
%%Order-theoretic notations
\newcommand{\maximum}[1]{\max\left(#1\right)}
\newcommand{\minimum}[2]{\min\left(#1\right)}
%%Probability-theoretic notations
\newcommand{\ind}[1]{\mathop{\mathrm{I}} \left\{ #1 \right\}} %indicator random variable
\newcommand{\pr}[1]{\mathop{\mathrm{Pr}} \left\{ #1 \right\}} %probability
\newcommand{\prc}[2]{\mathop{\mathrm{Pr}} \left\{ #1 \, \left| \, #2 \right\}\right.} %conditional prob.
\newcommand{\ex}[1]{\mathop{\mathrm{E}} \left[ #1 \right]} %expectation
\newcommand{\var}[1]{\mathop{\mathrm{Var}} \left[ #1 \right]} %variation

%New environment
\newcommand{\Input}{\item[Input:]} %ingredient of the new environment `problem'
\newcommand{\Output}{\item[Output:]} %ingredient of the new environment `problem'
\newenvironment{problem}{\begin{description}}{\end{description}} %specification of a computational problem
\newenvironment{loopinv}{\begin{quote}}{\end{quote}} %loop invariant
\newcommand{\Init}{\item[Initialization:]} %ingredient of the new environment `loopinvpf'
\newcommand{\Main}{\item[Maintenance:]} %ingredient of the new environment `loopinvpf'
\newcommand{\Term}{\item[Termination:]} %ingredient of the new environment `loopinvpf'
\newenvironment{loopinvpf}{\begin{description}}{\end{description}} %proof of loop invariant
\newtheorem*{remark}{Remark}

%Custom settings
\titleformat{\chapter}[hang]{\huge\bfseries}{\thechapter}{.5em}{}
\renewcommand{\thesubsection}{}

\begin{document}
%\noindent
%\textsc{\Huge Introduction to\\Algorithms}
%\\
%\textsc{\Large - Annotations and Solutions Manual -}
%\\
%\\
%\setcounter{chapter}{4}
\chapter{Probabilistic Analysis and Randomized Algorithms}
%Section 5.1-----------------------------------------------------------------------
\section{The hiring problem}
\subsection*{Exercises}
\begin{enumerate}[\thesection-1]
%
\item (Hint) Show the relation induced by the ranks on the candidates is a total order: reflexive, antisymmetirc, transitive, total.
%
\item (Hint) A number returned by $\proc{Random}(a, b)$ can be seen as $a$ plus a random number in the range $[0, b - a]$. Consider the binary representations of numbers in that range, or draw a binary tree for them.
\begin{remark}
What if $b - a$ is \emph{not} an exact power of $2$?
\end{remark}
%
\item (Hint) The idea is similar to the previous exercise.
%
\end{enumerate}
%End of Section 5.1----------------------------------------------------------------

%Section 5.2-----------------------------------------------------------------------
\section{Indicator random variables}
Skipped.
%End of Section 5.2----------------------------------------------------------------

%Problems of Chapter 5-------------------------------------------------------------
\section*{Problems}
\begin{enumerate}[\thechapter-1]
%
\item Skipped.
%
\end{enumerate}
\begin{appendices}
\setcounter{chapter}{1}
\chapter{Sets, Etc.}
%Section B.1----------------------------------------------------------------------------------
\section{Sets}
\subsection*{Exercises}
\begin{enumerate}[\thesection-1]
%
\item Skipped.
%
\item Use induction.
%
\item Use induction; in the induction step use the trick
\[
A_1 \union A_2 \union \etc \union A_n = (A_1 \union A_2 \union \etc \union A_{n - 1}) \union A_n.
\]
%
\item Note that the set of odd natural numbers is
\[
\setd{2n + 1}{n \in \nat}.
\]
%
\item Use induction on the cardinality of $S$; in the induction step consider a distinguished element.
%
\item Skipped.
%
\end{enumerate}
%End of Section B.1---------------------------------------------------------------------------

%Section B.2----------------------------------------------------------------------------------
\section{Relations}
\subsection*{Exercises}
\begin{enumerate}[\thesection-1]
%
\item Skipped.
%
\item Skipped.
%
\item \begin{enumerate}[a.]
%%
\item $\sete{1, 2, 3}^2 - \sete{(1, 3), (3, 1)}$.
%%
\item $<$ over $\nat$.
%%
\item The empty relation $\emptyset$ over $\nat$.
%%
\end{enumerate}
%
\item Possible typo: $S \cart S$ replaced by $S$.
%
\item No. Consider the empty relation $\emptyset$.
%
\end{enumerate}
%End of Section B.2---------------------------------------------------------------------------

%Section B.3----------------------------------------------------------------------------------
\section{Functions}
\subsection*{Exercises}
\begin{enumerate}[\thesection-1]
%
\item Note that in general, $\card{A} \geq \card{f(A)}$ and $f(A) \subseteq B$.
%
\item Skipped.
%
\item Skipped.
%
\item Use the bijections $f : \nat \to \zah$ and $\inv{f} : \zah \to \nat$ given in text, together with the bijection $g : \nat \to \nat \cart \nat$,
\[
g(m, n) = \frac{1}{2}(m + n)(m + n + 1) + m.
\]
%
\end{enumerate}
%End of Section B.3---------------------------------------------------------------------------
\end{appendices}
\end{document} 