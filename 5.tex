\setcounter{chapter}{4}
\chapter{Probabilistic Analysis and Randomized Algorithms}
%Section 5.1-----------------------------------------------------------------------
\section{The hiring problem}
\subsection*{Exercises}
\begin{enumerate}[\thesection-1]
%
\item Show the relation induced by the ranks on the candidates is a total order: reflexive, antisymmetirc, transitive, total.
%
\item A number returned by $\proc{Random}(a, b)$ can be seen as $a$ plus a random number in the range $[0, b - a]$. Consider the binary representations of numbers in that range, or draw a binary tree for them.
\begin{remark}
What if $b - a$ is \emph{not} an exact power of $2$?
\end{remark}
%
\item The idea is similar to the previous exercise.
%
\end{enumerate}
%End of Section 5.1----------------------------------------------------------------

%Section 5.2-----------------------------------------------------------------------
\section{Indicator random variables}
Skipped.
%End of Section 5.2----------------------------------------------------------------

%Problems of Chapter 5-------------------------------------------------------------
\section*{Problems}
\begin{enumerate}[\thechapter-1]
%
\item Skipped.
%
\end{enumerate}